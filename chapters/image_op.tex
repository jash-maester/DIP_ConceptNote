\chapter{Image Operations}

Image Operations are array operations.
These Operations are done on a pixel-by-pixel basis.
Array Operations are different from Matrix Operations.

Eg: Consider the two Images:
\begin{equation*}
    F_{1} \ =\ \begin{pmatrix}
    A & B\\
    C & D
    \end{pmatrix}
\end{equation*}
and
\begin{equation*}
    F_{2} \ =\ \begin{pmatrix}
    E & F\\
    G & H
    \end{pmatrix}
\end{equation*}

The multiplication of $F_1$ and $F_2$ is element-wise, as follows:

\begin{equation*}
    F_{1} \times F_{2}\ =\ \begin{pmatrix}
    AE & BF\\
    CG & DH
    \end{pmatrix}
\end{equation*}

In addition, one can observe that $F_1 \times F_2 = F_2 \times F_1$, whereas matrix multiplication is clearly different, since in matrices, $A \times B \neq B \times A$.
By default, image operations are array operations only.

\section{Arithmetic Operations}

Arithmetic operations include Image Addition, Subtraction, Multiplication, Division and Blending.

\subsection{Image Addition}

Two images can be added in a direct manner, as given by:
$$g(x,y) = f_1(x,y) + f_2(x,y)$$

The pixels of the input images $f_1(x,y)$ and $f_2(x,y)$ are added to obtain the resultant image $g(x,y)$.
However, it should be noted that the sum should not cross the allowed range for the datatype used.

Similarly it is possible to add any constant value to an image by:
$$g(x,y) = f_1(x,y) + k$$ where, $k$ is any constant.

If the value of $k$ is larger than 0, the overall `brightness' of the image is increased.

\textbf{Some Practical Applications of Image Addition:}
\subsubsection{Brightness of an image:}
The brightness of an image is the average pixel intensity of an image. If a positive or negative constant is added to all the pixels of an image, the average pixel intensity of the image increases or decreases, respectively. Thus adding a constant to all the pixels of an image increases its brightness and vice-versa. 

\subsubsection{Double Exposure:}
Double Exposure is the technique of superimposing an image on another image to produce the resultant. This gives the scenario equivalent to exposing a film to two pictures or Applying Watermarking on an image. This is done by adding One image on top of the other.

\subsection{Image Subtraction}

The Subtraction of two images can be done as follows:
$$g(x,y) = f_1(x,y) - f_2(x,y)$$
where, $f_1(x,y)$ and $f_2(x,y)$ are two input images and $g(x,y)$ is the output image.
To avoid negative values, it is desirable to find the modulus of the difference:
$$g(x,y) = |f_1(x,y) - f_2(x,y)|$$

It is also possible to subtract a constant value $k$ from the image, i.e. $g(x,y) = |f_1(x,y) - k|$, as $k$ is constant.
As discussed earlier, the decrease in the average intensity reduces the brightness of the image.

\textbf{Some practical applications of Image Subtraction are as follows:}
\begin{itemize}
    \item \textbf{Background Elimination:} This is done by subtracting the background image from the original image.
    \item \textbf{Brightness reduction:} This is done by subtracting a constant value from the whole image.
    \item \textbf{Change Detection:} This is done by subtracting two images and if the resultant value is non-zero, then there is a change between the two images.
\end{itemize}

\subsection{Image Multiplication}

Image Multiplication can be done in the following manner:
$$g(x,y) = f_1(x,y) \times f_2(x,y)$$
Here, $f_1(x,y)$ and $f_2(x,y)$ are two input images and $g(x,y)$ is the output image.

If the multiplied value crosses the maximum value of the data type of the images, the value of the pixel is reset to the maximum allowed value by that specific data type.

Similarly, scaling by a constant can be performed as:
$$g(x,y) = f(x,y) \times k$$
where, $k$ is a constant.

If $k$ is greater than 1, then the overall contrast increases.
If $k$ is less than 1, then the overall contrast decreases.


\textbf{NOTE:} The Brightness and Contrast can be manipulated together and also simulatneously by:
$$g(x,y) = a \times f(x,y) + k$$
Here, the parameters $a$ and $k$ are used to manipulate the contrast and brightness of the image $f(x,y)$ respectively, $g(x,y)$ is the output/resultant image.

\textbf{Some of the practical applications of Image Multiplication are as follows:}
\begin{itemize}
    \item \textbf{Manipulate Contrast:} If a fraction of $\geqslant$ 1 multiplied with the image, the overall contrast increases and vice-versa.
    \item \textbf{Filter Masks:} Useful for designing filter masks.
    \item \textbf{Highlighting AoI:} Used to create a mask to highlight area of interest.
\end{itemize}

\subsection{Image Division}

Similar to other previously discussed operations, the division operation can be performed in the following manner:
$$g(x,y) = \frac{f_1(x,y)}{f_2(x,y)}$$
where, $f_1(x,y)$ and $f_2(x,y)$ are the two input images and $g(x,y)$ is the output image.

Improper Data type specification may result in loss of information.

Division using a constant can also be performed, like:
$$g(x,y) = \frac{f(x,y)}{k}$$
where, $k$ is a constant.

\textbf{Some of the practical applications of Image Division are as follows:}
\begin{itemize}
    \item Change Detection.
    \item Separation of luminance and reflectance components.
    \item Contrast Reduction.
\end{itemize}

\section{Geometric Operations}
Let's discuss the Geometrical Operations used in Image Processing. These include: Translation, Scaling, Zooming, Linear Interpolation, Mirror or Reflection.

\subsection{Translation Operation}

Translation is the movement of an image to a new postion.
Let us assume that the point at the coordinate position $X = (x,y)$ of the matrix $F$ is moved to the new position $X'$ whose coordinate position is $(x',y')$.
Mathematically, this can be stated as a translation of a point $X$ to the new position $X'$.

This translation can be represented as:
\begin{gather*}
    x' = x + \delta x\\
    y' = y + \delta y
\end{gather*}

However, other transformations such as scaling and rotation are multiplicative in nature.
The transformation process for \textbf{rotation} is given by \textbf{$F' = RF$}, where \textbf{R} is the transform matrix for performing rotation.
Whereas, the transformation process for \textbf{scaling} is given by \textbf{$F' = SF$}. Here, \textbf{S} is the scaling transformation matrix.

To solve such a discrepancy and to create uniformity and consistency, it is necessary to use a homogeneous coordinate system where all transformations are treated as multiplications.
A point $(x,y)$ in 2D space is expressed as $(wx, wy, w)$ for $w \neq 0$.

The properties of homogeneous coordinates are as follows:
\begin{itemize}
    \item In homogeneous coordinates, at least one point should be non-zero. Thus $(0, 0, 0)$ does not exist in the homogeneous coordinate system.
    \item If one point is multiplicative of the other point, they are same. Thus, the points $(1, 3, 5)$ and $(3, 9, 15)$ are same as the second point is $3 \times (1, 3, 5)$.
    \item The point $(x, y, w)$ in the homogeneous coordinate system corresponds to the point $(\frac{x}{w}, \frac{y}{w})$ in 2D space.
\end{itemize}

Thus, in the homogeneous coordinate system, the translation process of the point $(x,y)$ to the new point $(x',y')$ of the image \textbf{$F$} is described as:
\begin{gather*}
    x' = x + \delta x\\
    y' = y + \delta y
\end{gather*}

In matrix form, this can be stated as:

\begin{equation*}
    \begin{bmatrix}
    x' & y' & 1
    \end{bmatrix}  = \begin{bmatrix}
    1 & 0 & \delta x\\
    0 & 1 & \delta y\\
    0 & 0 & 1
    \end{bmatrix}\begin{bmatrix}
    x & y & 1
    \end{bmatrix}^{T}
\end{equation*}

Sometimes, the image may not be present at the origin.
In that case, a suitable negative translational value can be used to bring the image to align with the origin.

\subsection{Scaling Operation}
Depending on the requirement, the object can be scaled.
\textit{Scaling} means enlarging and shrinking.
In the homogeneous coordinate system, the scaling of the point $(x,y)$ to the new point $(x',y')$ of the image \textbf{$F$} is described as:

\begin{gather*}
    x' = x \times S_x\\
    y' = y \times S_y
    \begin{bmatrix}
        x' & y'
    \end{bmatrix} = \begin{bmatrix}
        S_x & 0\\
        0 & S_y
    \end{bmatrix}\begin{bmatrix}
        x & y
    \end{bmatrix}
\end{gather*}

Here, $S_x$ and $S_y$ are called \textit{scaling factors} along $x$ and $y$ axes, respectively.
If the scaling factor is 1, then the object would appear larger.
If the scaling factors are fractions, the object would shrink.

\subsection{Zooming Operation}

The image can be zoomed using a process called pixel replication or interpolation.
Replication is called a zero-order hold process, where each pixel along the scan line is repeated once.
The the scan line is repeated.
The aim is to increase the number of pixels, thereby increasing the dimension of the image.

Eg: For the following image \textbf{$F$}:

\begin{equation}
    \begin{pmatrix}
        2 & 1\\
        1 & 3
    \end{pmatrix} = \begin{matrix}
        2 & 0 & 1 & 0\\
        0 & 0 & 0 & 0\\
        1 & 0 & 3 & 0\\
        0 & 0 & 0 & 0
    \end{matrix}
\end{equation}

This process is called \textbf{zero-order hold} process. Once 0s are inserted, the pixels are replicated to yield the following:

\begin{equation}
    \begin{matrix}
        2 & 2 & 1 & 1\\
        2 & 2 & 1 & 1\\
        1 & 1 & 3 & 3\\
        1 & 1 & 3 & 3
    \end{matrix}
\end{equation}

\subsection{Linear Interpolation Operation}

Consider the image:

\begin{equation*}
    H = \begin{pmatrix}
        2 & 1\\
        1 & 3
    \end{pmatrix}
\end{equation*}

Linear interpolation is equivalent to fitting a straight line by taking the average along the rows and columns.
The process is described as follows:
\begin{enumerate}
    \item For eg., the matrix $H$ can be zero-interlaced as:
        \begin{equation*}
            H = \begin{bmatrix}
                2 & 0 & 1 & 0\\
                0 & 0 & 0 & 0\\
                1 & 0 & 3 & 0\\
                0 & 0 & 0 & 0
            \end{bmatrix}
        \end{equation*}
    \item Interpolate the rows first. This is achieved by taking the average of the columns of that row. This yields:
        \begin{equation*}
            \begin{bmatrix}
                2 & 1.5 & 1 & 0.5\\
                0 & 0   & 0 & 0\\
                1 & 2   & 3 & 1.5\\
                0 & 0   & 0 & 0
            \end{bmatrix}
        \end{equation*}
    \item Interpolate Columns next. This is achieved by taking the average of the rows of that column. This yields:
        \begin{equation*}
            \begin{bmatrix}
                2   & 1.5  & 1   & 0.5\\
                1.5 & 1.75 & 2   & 1\\
                1   & 2    & 3   & 1.5\\
                0.5 & 1    & 1.5 & 0.75
            \end{bmatrix}
        \end{equation*}
\end{enumerate}

\subsection{Mirror or Reflection Operation}

This function creates the reflection of the object in a plane mirror. In other words, this function returns an image in which the pixels are reversed.
This operation is useful in creating an image in the desired order and for making comparisons.
The reflected object is of the same size as the original object, but the object is in the opposite quadrant.
Reflection is also described as rotation by $180\degree$.

The reflection along $x$-axis is given by:
$$F' = \begin{bmatrix}
    -x & y
\end{bmatrix} = \begin{bmatrix}
    1 & 0\\
    0 & -1
\end{bmatrix} \times \begin{bmatrix}
    x & y
\end{bmatrix}^{T}$$

Similarly, the reflection along the $y$-axis is given by:
$$F' = \begin{bmatrix}
    x & -y
\end{bmatrix} = \begin{bmatrix}
    -1 & 0\\
    0 & 1
\end{bmatrix} \times \begin{bmatrix}
    x & y
\end{bmatrix}^{T}$$

Similarly, the reflection along the $y = x$ is given by:
$$F' = \begin{bmatrix}
    x & -y
\end{bmatrix} = \begin{bmatrix}
    0 & 1\\
    1 & 0
\end{bmatrix} \times \begin{bmatrix}
    x & y
\end{bmatrix}^{T}$$

Similarly, the reflection along the $y=-x$ is given by:
$$F' = \begin{bmatrix}
    x & -y
\end{bmatrix} = \begin{bmatrix}
    0 & -1\\
    -1 & 0
\end{bmatrix} \times \begin{bmatrix}
    x & y
\end{bmatrix}^{T}$$


In homogeneous coordinate system, the matrices for reflection can be given as:
\begin{equation*}
    R_{y-axis} = \begin{bmatrix}
        -1 & 0 & 0\\
        0  & 1 & 0\\
        0  & 0 & 1
    \end{bmatrix} ;
    R_{x-axis} = \begin{bmatrix}
        1 & 0  & 0\\
        0 & -1 & 0\\
        0 & 0  & 1
    \end{bmatrix} ;
    R_{origin} = \begin{bmatrix}
        1 & 0 & 0\\
        0 & 1 & 0\\
        0 & 0 & -1
    \end{bmatrix}
\end{equation*}