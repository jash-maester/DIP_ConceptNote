\chapter{Image Enhancement}

\section{Histogram Manipulation}

Histogram Manipulation basically modifies the histogram of an input image so as to improve the visual quality of the image.
In order to understand histogram manipulation, it is necessary to understand what a `Histogram' is and how `histogram equalisation' technique affects the visual quality of the image.

\subsection{Histogram}
The Histogram of an image is a plot or visual representation of the number of occurences of each gray level present in the image.
The Histogram provides a convenient summary of the intensities of an image, but it is unable to convey any information regarding spatial relationships between individual pixels.
The histogram provides more insight about image \textit{contrast} and \textit{brightness}.

Some properties of a typical histogram are listed as follows:
\begin{enumerate}
    \item The histogram of a dark image will be clustered towards the lower gray levels.
    \item The histogram of a bright image will be clustered towards the higher gray levels.
    \item For a low-contrast image, the histogram will not be spread equally, i.e., the histogram will be narrow.
    \item For a high-contrast image, the histogram will have an equal spread in the gray levels.
\end{enumerate}

The Image Brightness may be improved by modifying the histogram of the image.

\subsection{Histogram Equalisation}

Equalisation is a process that attempts to spread out the gray levels in an image so that they are evenly distributed across their range.
It flattens the histogram to create a better quality image.
Histogram equalisation reassigns the brightness values of pixels based on the image histogram.
Histogram equalisation is a technique where the histogram of the resultant image is as flat as possible.
It also provides a more visually pleasing result across a wider range of images.

\subsubsection{Procedure to Perform Histogram Equalisation:}

Histogram equalisation is done by performing the following steps:
\begin{enumerate}
    \item Create the frequency distribution table for each gray level from the given input image. 
    \item Find the running sum of the histogram values.
    \item Normalise the values from the previous step, by dividing by the total number of pixels.
    \item Multiply the values from the previous step, by the maximum gray-level value and round it to nearest integer.
    \item Map the gray level values to the result obtained in the previous step, using a one-to-one correspondence.
\end{enumerate}

\subsubsection{Question:}
Perform histogram equalisation of the following image.
\begin{equation*}
    \begin{bmatrix}
        4 & 4 & 4 & 4 & 4\\
        3 & 4 & 5 & 4 & 3\\
        3 & 5 & 5 & 5 & 3\\
        3 & 4 & 5 & 4 & 3\\
        4 & 4 & 4 & 4 & 4
    \end{bmatrix}
\end{equation*}

\textit{Solution:}\ \ \ \ The maximum value in this case is found to be 5.
A minimum of 3 bits to represent the number.
There are eight possible gray levels from 0 to 7.
The histogram of the input image is given below:

\begin{table}[h!]
    \centering
    \begin{tabular}{|c|c|c|c|c|c|c|c|c|}
        \hline
        Gray Level & 0 & 1 & 2 & 3 & 4 & 5 & 6 & 7\\
        \hline
        Number of Pixels & 0 & 0 & 0 & 6 & 14 & 5 & 0 & 0\\
        \hline
    \end{tabular}
\end{table}

\textbf\textit{Step 1:} Compute the running sum of histogram values.

\begin{table*}[h!]
    \centering
    \begin{tabular}{|c|c|c|c|c|c|c|c|c|}
        \hline
        Gray Level & 0 & 1 & 2 & 3 & 4 & 5 & 6 & 7\\
        \hline
        Number of Pixels & 0 & 0 & 0 & 6 & 14 & 5 & 0 & 0\\
        \hline
        Running Sum & 0 & 0 & 0 & 6 & 20 & 25 & 25 & 25\\
        \hline
    \end{tabular}
\end{table*}

\textbf\textit{Step 2:} Divide the Running Sum obtained in \textit{Step 1} by the total number of pixels. In this case, the total number of pixels is \textbf{25}.

\begin{table*}[h]
    \centering
    \begin{tabular}{|p{4.5cm}|c|c|c|c|c|c|c|c|}
        \hline
        Gray Level & 0 & 1 & 2 & 3 & 4 & 5 & 6 & 7\\
        \hline
        Number of Pixels & 0 & 0 & 0 & 6 & 14 & 5 & 0 & 0\\
        \hline
        Running Sum & 0 & 0 & 0 & 6 & 20 & 25 & 25 & 25\\
        \hline
        Running Sum / Total Number of Pixels & 0/25 & 0/25 & 0/25 & 6/25 & 20/25 & 25/25 & 25/25 & 25/25\\
        \hline
    \end{tabular}
\end{table*}

\textbf\textit{Step 3:} Multiply the result obtained in \textit{Step 2} by the maximum gray-level value, which is 7 in this case.

\begin{table*}[h]
    \centering
    \begin{tabular}{|p{4cm}|c|c|c|c|c|c|c|c|}
        \hline
        Gray Level & 0 & 1 & 2 & 3 & 4 & 5 & 6 & 7\\
        \hline
        Number of Pixels & 0 & 0 & 0 & 6 & 14 & 5 & 0 & 0\\
        \hline
        Running Sum & 0 & 0 & 0 & 6 & 20 & 25 & 25 & 25\\
        \hline
        Running Sum & 0 & 0 & 0 & 6 & 20 & 25 & 25 & 25\\
        \hline
        Multiplying by Maximum Gray Level & $\frac{0}{25}\times 7$ & $\frac{0}{25}\times 7$ & $\frac{0}{25}\times 7$ & $\frac{6}{25}\times 7$ & $\frac{20}{25}\times 7$ & $\frac{25}{25}\times 7$ & $\frac{25}{25}\times 7$ & $\frac{25}{25}\times 7$\\
        \hline
    \end{tabular}
\end{table*}

\textit{Rounding off to the closest Integer values.}

\begin{table*}[h]
    \centering
    \begin{tabular}{|c|c|c|c|c|c|c|c|c|}
        \hline
        Gray Level & 0 & 1 & 2 & 3 & 4 & 5 & 6 & 7\\
        \hline
        Number of Pixels & 0 & 0 & 0 & 6 & 14 & 5 & 0 & 0\\
        \hline
        Running Sum & 0 & 0 & 0 & 6 & 20 & 25 & 25 & 25\\
        \hline
        Running Sum & 0 & 0 & 0 & 6 & 20 & 25 & 25 & 25\\
        \hline
        Multiplying by Maximum Gray Level & 0 & 0 & 0 & 2 & 6 & 7 & 7 & 7\\
        \hline
    \end{tabular}
\end{table*}

\textbf\textit{Step 4:} Mapping of Gray Levels by one-to-one correspondence.

\begin{table*}[h]
    \centering
    \begin{tabular}{|c|c|}
        \hline
        Original Gray Level & Histogram Equalised Values\\
        \hline
        0 & 0\\
        \hline
        1 & 0\\
        \hline
        2 & 0\\
        \hline
        3 & 2\\
        \hline
        4 & 6\\
        \hline
        5 & 7\\
        \hline
        6 & 7\\
        \hline
        7 & 7\\
        \hline
    \end{tabular}
\end{table*}

Here, the Original Image and Histogram Equalised Image are shown side by side:

\begin{equation*}
    \centering
    \begin{bmatrix}
        4 & 4 & 4 & 4 & 4\\
        3 & 4 & 5 & 4 & 3\\
        3 & 5 & 5 & 5 & 3\\
        3 & 4 & 5 & 4 & 3\\
        4 & 4 & 4 & 4 & 4
    \end{bmatrix} \rightarrow \begin{bmatrix}
        6 & 6 & 6 & 6 & 6\\
        2 & 6 & 7 & 6 & 2\\
        2 & 7 & 7 & 7 & 2\\
        2 & 6 & 7 & 6 & 2\\
        6 & 6 & 6 & 6 & 6
    \end{bmatrix}
\end{equation*}

\section{Bit-Plane Slicing}

The Gray Level of each pixel in a digital image is stored as one or more bytes in a computer.
For an 8-bit image, 0 is encoded as 0 0 0 0 0 0 0 0, and 255 is encoded as 1 1 1 1 1 1 1 1.
Any Number between 0 and 255 is encoded as one byte.
The bit in the far left side is referred as the Most Significant Bit (MSB), because a change in that bit would significantly change the value encoded by the byte.
The bit in the far right is referred as the Least Significant Bit (LSB), because a change in this bit does not change the encoded gray value much.

Bit plane slicing is a method of representing an image with one or more bits of the byte used for each pixel.
One can use only the MSB to represent a pixel, which reduces the original gray level to a binary image.
The three main goals of bit plane slicing are:
\begin{enumerate}
    \item Converting a gray level image to a bianry image.
    \item Representing an image with fewer bits and compressing the image to a smaller size.
    \item Enhancing the image by focusing.
\end{enumerate}

\subsection{Bit-Plane Transformations}

It is possible to transform a gray level image into a sequence of binary images by taking advantage of its bit pattern storage.
For example, a gray level image is stored as an 8-bit image. Therefore, the image can be transformed into an 8-level image where the zero planes consist of the last bit of each gray level.
The first plane consists of the first bit of each gray value.
Similarly, the most significant bit plane has the greatest effect in terms of the magnitude of the image.
It is roughly equivalent to the threshold of the image at level 127.
Depending on the requirement, the bit planes can be retained or ignored.

\subsubsection{Question:}
Show the bit-plane slicing of the following image:
\begin{equation*}
    \begin{bmatrix}
        7 & 6 & 5\\
        4 & 3 & 2\\
        1 & 1 & 0
    \end{bmatrix}
\end{equation*}

\textit{Solution:}
The Binary Equivalent of the pixels is:

\begin{equation*}
    \begin{bmatrix}
        111 & 110 & 101\\
        100 & 011 & 010\\
        001 & 001 & 000
    \end{bmatrix}
\end{equation*}

\textit{Case-I:} When LSB is changed to 0, this image is reduced to:

\begin{equation*}
    \begin{bmatrix}
        110 & 110 & 100\\
        100 & 010 & 010\\
        000 & 000 & 000
    \end{bmatrix}
\end{equation*}

The Equivalent Image of the above:

\begin{equation*}
    \begin{bmatrix}
        6 & 6 & 4\\
        4 & 2 & 2\\
        0 & 0 & 0
    \end{bmatrix}
\end{equation*}

\textit{Case-II:} When the MSB is changed to 0, this image changes to:

\begin{equation*}
    \begin{bmatrix}
        011 & 010 & 001\\
        000 & 011 & 010\\
        001 & 001 & 000
    \end{bmatrix}
\end{equation*}

The Equivalent Image of the above Binary Image:

\begin{equation*}
    \begin{bmatrix}
        3 & 2 & 1\\
        0 & 3 & 2\\
        1 & 1 & 0
    \end{bmatrix}
\end{equation*}

Thus, Bit-Plane Slicing Leads to Transformation of Images.